\documentclass[
  notoc % Suppress Tufte style table of contents.
]{tufte-book}

% Required Tufte packages.
\usepackage{changepage} % or changepage
\usepackage{fancyhdr}
\usepackage{fontenc}
\usepackage{geometry}
\usepackage{hyperref}
\usepackage{natbib}
\usepackage{bibentry}
\usepackage{optparams}
\usepackage{paralist}
\usepackage{placeins}
\usepackage{ragged2e}
\usepackage{setspace}
\usepackage{textcase}
\usepackage{textcomp}
\usepackage{titlesec}
\usepackage{titletoc}
\usepackage{xcolor}
\usepackage{xifthen}

\geometry{top=20mm,left=24mm,right=24mm,bottom=28mm}

% Tufte vs. Pandoc workaround.
% Issue: https://github.com/Tufte-LaTeX/tufte-latex/issues/64.
\renewcommand\allcapsspacing[1]{{\addfontfeature{LetterSpace=15}#1}}
\renewcommand\smallcapsspacing[1]{{\addfontfeature{LetterSpace=10}#1}}

% \setmainfont{TeX Gyre Pagella}
\usepackage[utf8]{inputenc}
\usepackage[T1]{fontenc}
\setmainfont{texgyrepagella}[
  Extension = .otf,
  UprightFont = *-regular,
  BoldFont = *-bold,
  ItalicFont = *-italic,
  BoldItalicFont = *-bolditalic,
]

\newfontfamily\JuliaMono{JuliaMono}[
  UprightFont = *-Regular,
  BoldFont = *-Bold
]
\newfontface\JuliaMonoRegular{JuliaMono-Regular}
\newfontface\JuliaMonoBold{JuliaMono-Bold}

\setmonofont{JuliaMono-Medium}[
  Contextuals=Alternate,
  Ligatures=NoCommon
]

\DeclareRobustCommand{\href}[2]{#2\footnote{\url{#1}}}

\usepackage{float}
\floatplacement{figure}{H}

% Listings Julia syntax definition.
\input{/home/runner/.julia/packages/Books/oARrL/defaults/julia_listings.tex}

% Unicode support.
\input{/home/runner/.julia/packages/Books/oARrL/defaults/julia_listings_unicode.tex}

% Used by Pandoc.
\providecommand{\tightlist}{%
  \setlength{\itemsep}{0pt}\setlength{\parskip}{0pt}
}
\newcommand{\passthrough}[1]{#1}

\usepackage{longtable}
\usepackage{booktabs}
\usepackage{array}

% Source: Wandmalfarbe/pandoc-latex-template.
\newlength{\cslhangindent}
\setlength{\cslhangindent}{1.5em}
\newlength{\csllabelwidth}
\setlength{\csllabelwidth}{3em}
\newenvironment{CSLReferences}[2] % #1 hanging-ident, #2 entry spacing
 {% don't indent paragraphs
  \setlength{\parindent}{0pt}
  % turn on hanging indent if param 1 is 1
  \ifodd #1 \everypar{\setlength{\hangindent}{\cslhangindent}}\ignorespaces\fi
  % set entry spacing
  \ifnum #2 > 0
  \setlength{\parskip}{#2\baselineskip}
  \fi
 }%
 {}
\usepackage{calc}
\newcommand{\CSLBlock}[1]{#1\hfill\break}
\newcommand{\CSLLeftMargin}[1]{\parbox[t]{\csllabelwidth}{#1}}
\newcommand{\CSLRightInline}[1]{\parbox[t]{\linewidth - \csllabelwidth}{#1}\break}
\newcommand{\CSLIndent}[1]{\hspace{\cslhangindent}#1}

\definecolor{linkblue}{HTML}{117af2}
\usepackage{hyperref}
\hypersetup{
  colorlinks,
  citecolor=linkblue,
  linkcolor=linkblue,
  urlcolor=linkblue,
  linktoc=page, % Avoid Table of Contents being nearly completely blue.
  pdftitle={A Researcher's Guide to Managing Knowledge},
  pdfauthor={Jacob S. Zelko},
  pdflang={en-US},
  breaklinks=true,
  pdfcreator={LaTeX via Pandoc}%
}
\urlstyle{same} % disable monospaced font for URLs

\title{A Researcher's Guide to Managing Knowledge}
\author{\noindent{Jacob S. Zelko}\\[3mm] }
\date{}

% Re-enable section numbering which was disabled by tufte.
\setcounter{secnumdepth}{2}

% Fix captions for longtable.
% Thanks to David Carlisle at https://tex.stackexchange.com/a/183344/92217.
\makeatletter
\def\LT@makecaption#1#2#3{%
  \noalign{\smash{\hbox{\kern\textwidth\rlap{\kern\marginparsep
  \parbox[t]{\marginparwidth}{\vspace{12pt}%
\@tufte@caption@font \@tufte@caption@justification \noindent
   #1{#2: }\ignorespaces #3}}}}}}
\makeatother

% Doesn't seem to do anything.
\usepackage{float}
\floatplacement{figure}{H}
\floatplacement{table}{H}

% Reduce large spacing around sections.
\titlespacing*{\chapter}{0pt}{5pt}{20pt}
\titlespacing*{\section}{0pt}{2.5ex plus 1ex minus .2ex}{1.3ex plus .2ex}
\titlespacing*{\subsection}{0pt}{1.75ex plus 1ex minus .2ex}{1.0ex plus.2ex}

\titleformat{\chapter}%
  [hang]% shape
  {\normalfont\huge\itshape}% format applied to label+text
  {\huge\thechapter}% label
  {1em}% horizontal separation between label and title body
  {}% before the title body
  []% after the title body

% Reduce spacing in table of contents.
\usepackage{etoolbox}
\makeatletter
\pretocmd{\chapter}{\addtocontents{toc}{\protect\addvspace{-3\p@}}}{}{}
\pretocmd{\section}{\addtocontents{toc}{\protect\addvspace{-4\p@}}}{}{}
\pretocmd{\subsection}{\addtocontents{toc}{\protect\addvspace{-5\p@}}}{}{}
\makeatother

% Long texts are harder to read than tables.
% Therefore, we can reduce the font size of the table.
\AtBeginEnvironment{longtable}{\footnotesize}

% Some space between paragraphs is necessary because code blocks can output single line paragraphs.
\setlength\parskip{1em plus 0.1em minus 0.2em}

% For justified text.
\usepackage{ragged2e}

% tufte-book disables subsubsections by default.
% Got this definition back via `\show\subsubsection`.

\usepackage{amsfonts}
\usepackage{amssymb}
\usepackage{amsmath}
\usepackage{unicode-math}

% URL line breaks.
\usepackage{xurl}

% Probably doesn't hurt.
\usepackage{marginfix}


\begin{document}

\makeatletter
\thispagestyle{empty}
\vfill
{\Huge\bf
\noindent
\@title
}\\[1in]
{\Large
\noindent
\@author
}
\makeatother

\makeatletter
\newpage
\thispagestyle{empty}
\vfill
{\noindent

}
\vfill
{\small


2021-11-10

All Rights Reserved. Copyright ©2021
}
\makeatother


% Don't remove this or authors will show up in header of every page.
\frontmatter
\mainmatter
\fancyfoot[C]{All Rights Reserved. Copyright ©2021}

\setcounter{tocdepth}{1}
\tableofcontents

% Justify text.
\justifying

% parindent seems to be set from within another class too.
% it is really not useful here because it will also indent lines directly after
% code blocks. Which most of the times not useful.
\setlength{\parindent}{0pt}

\hypertarget{why-this-book}{%
\chapter{Why This Book? 📖}\label{why-this-book}}

I did not intend to write this book. Yet, here I am, straddling a log
above a brook log listening to the soothing symphony of the running
water juxtaposed against the harsh whir of cars on the main roadway not
more than a rocks throw from where I precariously sit. And I suppose,
that this description is a very apt analogy for how I found myself
writing this book.

Away from the wooded groves and busy streets, for a majority of my life
I have found myself not enjoying the vivacious odor of leaves and pines,
but musty books and freshly steeped tea. As both a professional student
and at-times independent, at-times employed researcher, I have observed
an interesting phenomena that can best be characterized as
``intellectual waste.'' What I mean by this is that I have the great
fortune and privilege of being frequently surrounded by brilliant
people. Medical doctors, lawyers, company founders - I often interact
and even share the occasional coffee or cocoa with stunning minds such
as these. Although brilliant, a common observation I have made amongst
these professionals from my perch on my proverbial ``log'' is the lack
of intellectual organization.

I vividly remember one conversation with a colleague of mine - a
prolific mathematician turned bioninformatician - on the matter of note
taking during literature reviews. As a rather novice, though not
inexperienced, researcher at the time, I held great respect for this
colleague's thoughts and was eager to hear how he conducted his work.
The colleague opened a directory on his personal computer and showed me
a plethora of different sub-directories, each containing notes and
thoughts on particular papers. It was no shock that it contained so many
notes, but I couldn't help but wonder: ``How could they find anything
when they needed it?'' When I asked about how easy it was to find a
particular fact or piece information from this system, the answer was
given with a sheepish grin: ``Not very.'' That question was the catalyst
which spurred me to take the question of how to reduce intellectual
waste very seriously - especially as I began my own journey as a
researcher.

Before you read another word, let me lay out exactly what this book will
discuss - and more importantly, what it \emph{won't}.

\hypertarget{who-is-this-book-for}{%
\section{Who Is This Book For?}\label{who-is-this-book-for}}

Foo Bar

\hypertarget{who-is-this-book-not-for}{%
\section{Who Is This Book Not For?}\label{who-is-this-book-not-for}}

Foo Baz

\hypertarget{knowledge-building}{%
\chapter{Knowledge Building 📚}\label{knowledge-building}}

This chapter concerns what I have come across in my educational
practices as being of extreme use to me as a researcher and student. The
information provided here encapsulates a variety of information and
techniques one can make use of in their studies or research. Here are
the executive summaries of each section:

\textbf{\ref{sec:screening-heuristics} Screening Heuristics} - these are
some of my own heuristics I use when scanning media. I encourage you to
let these inspire you in creating your own heuristics!

\textbf{\ref{sec:spacing-effect} The Spacing Effect} - the spacing
effect is a useful way to schedule your learning for maximum retention
and understanding of material.

\textbf{\ref{sec:spaced-repetition-systems} Spaced Repetition Systems} -
modern tools and suggestions on how to incorporate the spacing effect
into your learning.

\textbf{\ref{sec:incremental-reading} Incremental Reading} - a paradigm
to read multiple pieces of information at once to maximize reading while
learning.

These sections expose you to many ideas that can help with you
developing your own learning process. Later chapters go more into how to
connect all of this together in a more practical manner.

\hypertarget{sec:screening-heuristics}{%
\section{Screening Heuristics}\label{sec:screening-heuristics}}

When faced with a piece of media to consume, I apply the following
heuristics to determine if it is worth my time:

\begin{enumerate}
\def\labelenumi{\arabic{enumi}.}
\item
  \textbf{Do I want to consume this?} Often, I open media after reading
  the title and sub-heading. If it sounds interesting, I might process
  it, if not: pass.
\item
  \textbf{Am I motivated to process this?} If I am not motivated to
  process a piece of media, I either throw it out or save it for
  processing later. If I keep it and have not processed it after some
  arbitrary time, I don't bother processing it and throw it out.
\item
  \textbf{Do I find the main thesis of the piece intriguing?} I will
  jump from the beginning to the end of a piece. If the thesis seems
  intriguing to me I will investigate the conclusion. If the ending
  convincingly ties itself back to the thesis, I deem it worth my time.
  If it fails any of these moments, I throw it out.
\end{enumerate}

To qualify these heuristics, an explanation. I operate under the
implicit assumption that time is scarce. Every thing wants your time.
Therefore, it is imperative to triage valuable information from noise. I
rationalize that the cost of missing something in my screening process
is low. If a piece is worth my time, I will be led back to it.

Furthermore, I am a selfish in my consumption of media. I choose to
consume that which I wish to consume. While processing a piece of media,
I may skip around in it until I find something interesting.

\hypertarget{sec:spacing-effect}{%
\section{The Spacing Effect}\label{sec:spacing-effect}}

The spacing effect accounts for the fact that learning is improved when
studying is spread out over time. It was first discovered and described
by Hermann Ebbinghaus during 1880 - 1885.
(\protect\hyperlink{ref-ebbinghaus1885ueber}{Ebbinghaus, 1885}) His
findings were later accurately reproduced and documented by \emph{Murre
\& Dros}.
(\protect\hyperlink{ref-murreReplicationAnalysisEbbinghaus2015}{Murre \&
Dros, 2015}) \href{05282020172154-replication-ebbinghaus.md}{(Click for
results of that study here)}

The spacing effect has not been very well utilized in US-based education
institutes despite its multitudinous benefits.
(\protect\hyperlink{ref-dempsterCaseStudyFailure1988}{Dempster, 1988})
Cepeda et al.~determined that in a memory based challenge, spaced based
learning outperformed massed learning \textasciitilde96\% of the time.
(\protect\hyperlink{ref-cepeda2006distributed}{Cepeda et al., 2006})
This coincides with the deficient processing view posited by Hintzman
that stated how massed repetition leads to a lack of attention in later
reviews. (\protect\hyperlink{ref-hintzman1974theoretical}{Hintzman,
1974})

According to \emph{Pyc and Rawson}, labored but correct recall while
practicing improves memory. Spacing items of recall produce greater
effort during retrieval and enables thorough conversion of the item to
memory.\footnote{Testing} (\protect\hyperlink{ref-pyc2009testing}{Pyc \&
Rawson, 2009}) Semantic processing of information during repetitions
assists in making that information more
\href{03172020033742-antifragility.md}{anti-fragile, to borrow from
Nassim Taleb,} during reviews. This causes performance in later memory
testing to be unaffected by changes in such things like the type of font
used when presented information.\footnote{Testing}
(\protect\hyperlink{ref-mammarella2002spacing}{Mammarella et al., 2002})

A confounding factor in the idea of the spacing effect is the encoding
variability theory. This theory states that one's performance on a
memory test is related to overlaps amongst current contextualized
material both during testing and while encoding. According to this view,
spaced repetition typically entails some variability in presentation
contexts. Yet, this results in a positive outcome being that there are
then more retrieval cues associated with that material.
(\protect\hyperlink{ref-cormier2014basic}{Cormier, 2014})

However, there are concerns about the spacing effect that have impeded
its overall adoption into educational formats. \emph{Dempster} made a
case examining potential rationales for the lack of adoption in a review
of the current state of the spacing effect (though the study is old over
30 years old, it still remains that education systems do not incorporate
the spacing effect).
(\protect\hyperlink{ref-dempsterCaseStudyFailure1988}{Dempster, 1988})
His biggest point of concern was the lack of studies that showed
effective classroom utilization of the method.

\hypertarget{sec:spaced-repetition-systems}{%
\section{Spaced Repetition
Systems}\label{sec:spaced-repetition-systems}}

Despite such misgivings, it is still a phenomenon that have given rise
to many benefits and potential applications for at least personal
education. Principle of which is in the form of
\href{05252020183020-spaced-repetition-systems.md}{spaced repetition
systems.} These systems have determined the best spacing algorithms for
a learner to use the spacing effect to assist in learning diverse
educational material.

There are multiple algorithms available used for effectively
implementing the spacing effect. The earliest algorithm comes from the
original discoverer of the spacing effect, Hermann Ebbinghaus. The
algorithm \href{05282020172154-replication-ebbinghaus.md}{has been
replicated} and scrutinized examined multiple times.
(\protect\hyperlink{ref-bjork2011making}{Bjork et al., 2011};
\protect\hyperlink{ref-ebbinghaus1885ueber}{Ebbinghaus, 1885};
\protect\hyperlink{ref-murreReplicationAnalysisEbbinghaus2015}{Murre \&
Dros, 2015}) Spaced repetition systems are based around proper
implementations of such algorithms.

Before the advent of computers, it was very difficult to implement such
a system. The best system to take advantage of this effect was the
\href{05102020220941-leitner-system.md}{Leitner System which used paper
note cards} to assist with remembering information.
(\protect\hyperlink{ref-sebastianleitnerLerntManLernen1972}{Sebastian
Leitner, 1972}) However, this was still a somewhat fragile system and
difficult to manage.

With computers, there have been several pieces of software made to
automate and keep track of repetitions \emph{for a learner}. A majority
of these systems directly adapted the Leitner System from the analog to
the digital world. Each software generally has proprietary algorithms
that supposedly maximize the spacing effect but are directly inspired by
the original work done by Ebbinghaus.

Whether analog or digital, key features of these systems are as follows:

\begin{enumerate}
\def\labelenumi{\arabic{enumi}.}
\tightlist
\item
  Uses an algorithm implementation of the spacing effect for a user.
\item
  Each piece of information is spaced automatically per a user's
  perceived difficulty in either remembering or processing that piece of
  information on a repetition.
\item
  Pieces of information are generally presented in a ``flashcard''
  format and are stored in ``decks.''
\end{enumerate}

\hypertarget{sec:incremental-reading}{%
\section{Incremental Reading}\label{sec:incremental-reading}}

Piotr Woźniak, a Polish learning researcher, is thought to be the
originator of the term and process of ``incremental reading.''
Incremental reading is a methodology that enables one to process
multiple written pieces concurrently over several days while promoting
learning. The methodology follows this general format:

\begin{enumerate}
\def\labelenumi{\arabic{enumi}.}
\tightlist
\item
  Identify written pieces of interest.
\item
  Prioritize the pieces according to personal urgency. Adjust while
  reading.
\item
  Allocate a comfortable reading time duration per piece with a timed,
  short break in-between.\footnote{Testing}
\item
  While reading each piece, annotate material that is of interest or
  difficult to process.
\item
  Transfer annotations from each piece to an index card representation
  (actual or digital).
\item
  Pause reading and review collected annotations using a spaced
  repetition system.
\item
  Once annotations have been thoroughly revised and reviewed, repeat
  from step 2 until finished with pieces over the next several days.
\end{enumerate}

Incremental reading aids attention and encourages better retention via
intermittent breaks and mixing of reading materials,
respectively.\footnote{Testing}
(\protect\hyperlink{ref-arigaBriefRareMental2011}{Ariga \& Lleras,
2011}; \protect\hyperlink{ref-bjork2011making}{Bjork et al., 2011};
\protect\hyperlink{ref-shea1979contextual}{Shea \& Morgan, 1979})
Ingesting raw annotations into a spaced repetition system makes
processing them manageable as the cards they are written appear at
spaced intervals. On each appearance of a card, one can rewrite or
modify that card; further cementing it in memory.\footnote{Testing}
(\protect\hyperlink{ref-bjork2011making}{Bjork et al., 2011}) After
finishing revisions and reviews, this general process is repeated until
done with reading the pieces.

\hypertarget{researching-courageously}{%
\chapter{Researching Courageously ⚔️}\label{researching-courageously}}

\begin{quote}
Once you get your courage up and believe that you can do important
problems, then you can. - Richard Hamming, ``You and Your Research''
\end{quote}

\hypertarget{what-is-great}{%
\section{What Is ``Great?''}\label{what-is-great}}

Based off the talk by Richard Hamming, \emph{``You and Your Research''}.

\hypertarget{asking-great-questions}{%
\section{Asking Great Questions}\label{asking-great-questions}}

The art of making great research.

\hypertarget{great-ideas}{%
\section{Great Ideas}\label{great-ideas}}

\hypertarget{creating-research-directions}{%
\section{Creating Research
Directions}\label{creating-research-directions}}

The distinguishing feature between research directions and research
ideas is a matter of pinning. Generally, up until now, the ideation is
fantastic. It is playful, it is creative, and it is unfettered by any
barrier.

A research direction however, requires one to define what it is they
want to pursue. An idea is only an idea, until it is succinctly
communicated. That communication, too early, can be a negative barrier
to creativity as one may not have had enough time to play with the
thoughts. It is a balancing act of determining at what phase an idea is
ready to be communicated.

My personal rule of thumb for this is as follows:

\begin{enumerate}
\def\labelenumi{\arabic{enumi}.}
\item
  Have an idea and explore it thoroughly through whatever means
  suitable.
\item
  Work towards having a mental sketch of what this idea could lead to.
\item
  When you start feeling like you want to share the idea with someone,
  share it.
\end{enumerate}

\hypertarget{funding-your-ideas}{%
\chapter{Funding Your Ideas 💰}\label{funding-your-ideas}}

Grants. For some, the word brings excitement. Others: terror.

\hypertarget{before-your-funding-search}{%
\section{Before Your Funding Search}\label{before-your-funding-search}}

\hypertarget{section}{%
\section{}\label{section}}

\hypertarget{appendix}{%
\chapter*{Appendix}\label{appendix}}
\addcontentsline{toc}{chapter}{Appendix}

Foo bar

\hypertarget{references}{%
\chapter*{References}\label{references}}
\addcontentsline{toc}{chapter}{References}

\hypertarget{refs}{}
\begin{CSLReferences}{1}{0}
\leavevmode\hypertarget{ref-arigaBriefRareMental2011}{}%
Ariga, A., \& Lleras, A. (2011). Brief and rare mental {``breaks''} keep
you focused: {Deactivation} and reactivation of task goals preempt
vigilance decrements. \emph{Cognition}, \emph{118}(3), 439--443.
\url{https://doi.org/10.1016/j.cognition.2010.12.007}

\leavevmode\hypertarget{ref-bjork2011making}{}%
Bjork, E. L., Bjork, R. A., \& others. (2011). Making things hard on
yourself, but in a good way: {Creating} desirable difficulties to
enhance learning. \emph{Psychol. Real World Essays Illus. Fundam.
Contrib. Soc.}, \emph{2}(59-68).

\leavevmode\hypertarget{ref-cepeda2006distributed}{}%
Cepeda, N. J., Pashler, H., Vul, E., Wixted, J. T., \& Rohrer, D.
(2006). Distributed practice in verbal recall tasks: {A} review and
quantitative synthesis. \emph{Psychol. Bull.}, \emph{132}(3), 354.

\leavevmode\hypertarget{ref-cormier2014basic}{}%
Cormier, S. M. (2014). \emph{Basic processes of learning, cognition, and
motivation}. {Psychology Press}.

\leavevmode\hypertarget{ref-dempsterCaseStudyFailure1988}{}%
Dempster, F. N. (1988). A {Case Study} in the {Failure} to {Apply} the
{Results} of {Psychological Research}. \emph{Am. Psychol.}, 8.

\leavevmode\hypertarget{ref-ebbinghaus1885ueber}{}%
Ebbinghaus, H. (1885). \emph{Ueber das gedächtnis}.

\leavevmode\hypertarget{ref-hintzman1974theoretical}{}%
Hintzman, D. L. (1974). \emph{Theoretical implications of the spacing
effect.}

\leavevmode\hypertarget{ref-mammarella2002spacing}{}%
Mammarella, N., Russo, R., \& Avons, S. (2002). Spacing effects in
cued-memory tasks for unfamiliar faces and nonwords. \emph{Mem.
Cognit.}, \emph{30}(8), 1238--1251.

\leavevmode\hypertarget{ref-murreReplicationAnalysisEbbinghaus2015}{}%
Murre, J. M. J., \& Dros, J. (2015). Replication and {Analysis} of
{Ebbinghaus}' {Forgetting Curve}. \emph{PLoS ONE}, \emph{10}(7),
e0120644. \url{https://doi.org/10.1371/journal.pone.0120644}

\leavevmode\hypertarget{ref-pyc2009testing}{}%
Pyc, M. A., \& Rawson, K. A. (2009). Testing the retrieval effort
hypothesis: {Does} greater difficulty correctly recalling information
lead to higher levels of memory? \emph{J. Mem. Lang.}, \emph{60}(4),
437--447.

\leavevmode\hypertarget{ref-sebastianleitnerLerntManLernen1972}{}%
Sebastian Leitner. (1972). \emph{So lernt man lernen: {Aangewandte
Lernpsychologie} - ein {Weg} zum {Erfolg}}. {Verlag Herder}.

\leavevmode\hypertarget{ref-shea1979contextual}{}%
Shea, J. B., \& Morgan, R. L. (1979). Contextual interference effects on
the acquisition, retention, and transfer of a motor skill. \emph{J. Exp.
Psychol. {[}Hum. Learn.{]}}, \emph{5}(2), 179.

\end{CSLReferences}

\backmatter

\end{document}
